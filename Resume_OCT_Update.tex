

\documentclass[letterpaper,11pt]{article}

\usepackage{latexsym}
\usepackage[empty]{fullpage}
\usepackage{titlesec}
\usepackage{marvosym}
\usepackage[usenames,dvipsnames]{color}
\usepackage{verbatim}
\usepackage{enumitem}
\usepackage[hidelinks]{hyperref}
\usepackage{fancyhdr}
\usepackage[english]{babel}
\usepackage{tabularx}
\input{glyphtounicode}


%----------FONT OPTIONS---------
% sans-serif
% \usepackage[sfdefault]{FiraSans}
% \usepackage[sfdefault]{roboto}
% \usepackage[sfdefault]{noto-sans}
% \usepackage[default]{sourcesanspro}

% serif
% \usepackage{CormorantGaramond}
% \usepackage{charter}


\pagestyle{fancy}
\fancyhf{} % clear all header and footer fields
\fancyfoot{}
\renewcommand{\headrulewidth}{0pt}
\renewcommand{\footrulewidth}{0pt}

% Adjust margins
\addtolength{\oddsidemargin}{-0.5in}
\addtolength{\evensidemargin}{-0.5in}
\addtolength{\textwidth}{1in}
\addtolength{\topmargin}{-.5in}
\addtolength{\textheight}{1.0in}

\urlstyle{same}

\raggedbottom
\raggedright
\setlength{\tabcolsep}{0in}

% Sections formatting
\titleformat{\section}{
  \vspace{-4pt}\scshape\raggedright\large
}{}{0em}{}[\color{black}\titlerule \vspace{-5pt}]

% Ensure that generate pdf is machine readable/ATS parsable
\pdfgentounicode=1

%-----------------------
% Custom commands
\newcommand{\resumeItem}[1]{
  \item\small{
    {#1 \vspace{-2pt}}
  }
}

\newcommand{\resumeSubheading}[4]{
  \vspace{-2pt}\item
    \begin{tabular*}{0.97\textwidth}[t]{l@{\extracolsep{\fill}}r}
      \textbf{#1} & #2 \\
      \textit{\small#3} & \textit{\small #4} \\
    \end{tabular*}\vspace{-7pt}
}

\newcommand{\resumeSubSubheading}[2]{
    \item
    \begin{tabular*}{0.97\textwidth}{l@{\extracolsep{\fill}}r}
      \textit{\small#1} & \textit{\small #2} \\
    \end{tabular*}\vspace{-7pt}
}

\newcommand{\resumeProjectHeading}[2]{
    \item
    \begin{tabular*}{0.97\textwidth}{l@{\extracolsep{\fill}}r}
      \small#1 & #2 \\
    \end{tabular*}\vspace{-7pt}
}

\newcommand{\resumeSubItem}[1]{\resumeItem{#1}\vspace{-4pt}}

\renewcommand\labelitemii{$\vcenter{\hbox{\tiny$\bullet$}}$}

\newcommand{\resumeSubHeadingListStart}{\begin{itemize}[leftmargin=0.15in, label={}]}
\newcommand{\resumeSubHeadingListEnd}{\end{itemize}}
\newcommand{\resumeItemListStart}{\begin{itemize}}
\newcommand{\resumeItemListEnd}{\end{itemize}\vspace{-5pt}}

%------------------------------------------
%%%%%%  RESUME STARTS HERE  %%%%%%%%%%%%%%%%%%%%%%%%%%%%


\begin{document}



\begin{center}
    \textbf{\Huge \scshape Manoj Kumar Ashok} \\ \vspace{1pt}
    \small (312)-284 9898 $|$ \href{mailto:mashok@depaul.edu}{\underline{Mashok@depaul.edu}} $|$ 
    \href{https://www.linkedin.com/in/manoj-kumar-ashok-078241211/}{\underline{linkedin.com/Manoj}} $|$
    \href{https://github.com/MK-Github03}{\underline{github.com/Manoj}} 
\end{center}


%-----------EDUCATION-----------
\section{Education}
  \resumeSubHeadingListStart
 
    \resumeSubheading
      {DePaul University}{Chicago, IL}
      {Master of Data Science, Concentrated on computational methods - \textbf{GPA:} 4.0/4.0}{Jan 2024 -- Nov 2025  }
       \vspace{0.1em}
      
      {\textbf{Coursework}: Data analysis and regression, \textbf{Mining Big data}, Advanced Machine Learning, Fundamentals of Data Science, Advanced Data analysis, Database processing for large scale analytics, \textbf{Neural Networks and Deep Learning}, \textbf{Natural Language Processing}. }
    
  \resumeSubHeadingListEnd
%-----------PROGRAMMING SKILLS-----------
\section{Technical Skills}
 \begin{itemize}[leftmargin=0.15in, label={}]
    \small{\item{
     \textbf{Languages}{: Python, R, SQL} \\
     \textbf{Frameworks}{: React, Node.js, Flask, JUnit, WordPress, Material-UI, FastAPI} \\
     \textbf{Developer Tools}{: RStudio, SQL Server, Git, hadoop, Apache spark, AWS-redshift, Azure, Big Query, Dataflow,} \\
     \textbf{Database Management}{: Database Management: MySQL, RDBMS – Oracle, ETL, Data Pipelines, Airflow, SSIS }
     \textbf{Visualization}{: Matplotlib, Seaborn, ggplot2, Tableau, PowerBI, quickSight, snowflake }
    }}
 \end{itemize}


%-----------EXPERIENCE-----------
\section{Experience}
  \resumeSubHeadingListStart

    \resumeSubheading
      {Software Developer}{\textit{Jan 2023--July 2023}}
      {Zoho Corporation }{Chennai, India}
      \resumeItemListStart
        \resumeItem{\textbf{Developed and optimized data processing} scripts using \textbf{Python} and \textbf{SQL} to automate the \textbf{extraction, transformation}, and analysis of large datasets from Zoho CRM and Zoho Books, resulting in a \textbf{20\%} increase in \textbf{data management} efficiency.}
        \resumeItem{Engineered custom solutions for data visualization and reporting using \textbf{Tableau }and\textbf{ Microsoft Excel}, integrating advanced \textbf{SQL queries} and automating \textbf{data pipelines}, which improved business process insights and decision-making.}
        \resumeItem{Designed and implemented \textbf{data models} and \textbf{statistical analysis} workflows to track key trends in support data, improving customer issue resolution by \textbf{10\%} through more efficient querying and reporting.}
        \resumeItem{Collaborated with cross-functional teams to architect and deploy scalable \textbf{ETL pipelines} using \textbf{Airflow} and optimize \textbf{data processing} workflows, reducing processing time by \textbf{15\%} and ensuring reliable data flow across systems.}
      \resumeItemListEnd
      
% -----------Multiple Positions Heading-----------
%    \resumeSubSubheading
%     {Software Engineer I}{Oct 2014 - Sep 2016}
%     \resumeItemListStart
%        \resumeItem{Apache Beam}
%          {Apache Beam is a unified model for defining both batch and streaming data-parallel processing pipelines}
%     \resumeItemListEnd
%    \resumeSubHeadingListEnd
%-------------------------------------------
%for subheading and some examples
%\resumeSubheading
%   {Position Title}{Date Range}
  %    {Organization}{Location}
  %    \resumeItemListStart
  %      \resumeItem{Description of responsibility or achievement}
  %%      \resumeItem{Description of responsibility or achievement}
   %     \resumeItem{Description of responsibility or achievement}
   % \resumeItemListEnd

% \resumeSubheading
%     {Position Title}{Date Range}
 %     {Organization}{Location}
  %%    \resumeItemListStart
   %     \resumeItem{Description of responsibility or achievement}
      %  \resumeItem{Description of responsibility or achievement}
     %   \resumeItem{Description of responsibility or achievement}
     %   \resumeItem{Description of responsibility or achievement}
     %   \resumeItem{Description of responsibility or achievement}
     %   \resumeItem{Description of responsibility or achievement}
     % \resumeItemListEnd


%  \resumeSubHeadingListEnd


%-----------PROJECTS-----------
\section{Projects}
    \resumeSubHeadingListStart
      \resumeProjectHeading
          {\textbf{\underline{\href{https://github.com/MK-Github03/Speech-Emotion_recognition-using-Deep-learning}{Speech Emotion Detection using Deep Learning}}} $|$ \emph{\textbf{\scriptsize  TensorFlow, \scriptsize SVM,\scriptsize NumPy, \scriptsize Matplotlib}}}{\textit{July 2023}}
          \resumeItemListStart
          \vspace{0.5em} % Adds vertical space of 0.5em
            \resumeItem{Developed a Speech Emotion Detection model in \textbf{Python}, achieving \textbf{85\%} accuracy using \textbf{TensorFlow} and \textbf{SVM}, with audio feature extraction performed by LibROSA (e.g., MFCC, Chroma, and Mel Spectrogram).}
            \resumeItem{Built and optimized data pipelines using \textbf{Git} and \textbf{NumPy}, reducing preprocessing time by \textbf{30\%} and improving data handling efficiency by \textbf{40\%}.}
            \resumeItem{Enhanced emotion classification accuracy by \textbf{25\%} through advanced feature engineering and hyperparameter tuning techniques using \textbf{Scikit-Learn}.}
            \resumeItem{Visualized model performance and insights using \textbf{Matplotlib}, increasing the efficiency of data interpretation by \textbf{20\%}}
          \resumeItemListEnd
      \resumeProjectHeading
          {\textbf{Predictive Analysis for credit limit} $|$ \emph{\textbf{\scriptsize Python,\scriptsize Scikit-Learn,\scriptsize TensorFlow,\scriptsize SQL,\scriptsize Pandas,\scriptsize NumPy,}}}{\textit{July 2024}}
          \resumeItemListStart
           \vspace{0.5em}
            \resumeItem{Developed machine learning model using \textbf{Python}, \textbf{Scikit-Learn}, and TensorFlow, improving prediction accuracy by \textbf{20\%.} Applied \textbf{Ridge} and \textbf{Lasso} Regression to reduce overfitting and improve model generalization.}
            \resumeItem{Preprocessed data with \textbf{Pandas} and \textbf{NumPy}, reducing anomalies by \textbf{25\%} and increasing model accuracy by \textbf{18\%} through feature scaling, outlier detection, and data normalization.}
            \resumeItem{Used advanced \textbf{feature engineering} and hyperparameter tuning, boosting model performance by \textbf{22\%} and validating results through \textbf{cross-validation}. Evaluated models using \textbf{RMSE, MAE, and R²}.}
            \resumeItem{Enhanced \textbf{data pipeline} efficiency with \textbf{SQL}, streamlining data extraction for scalable machine learning pipelines, and visualized insights with \textbf{Matplotlib} for effective communication.}
          \resumeItemListEnd
    \resumeSubHeadingListEnd




%


%-------------------------------------------
\end{document}
